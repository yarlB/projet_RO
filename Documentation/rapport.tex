B1;2802;0c\documentclass[a4paper,8pt]{report}
\usepackage[francais]{babel}
\usepackage[utf8]{inputenc}
\usepackage[T1]{fontenc}
\usepackage{amsmath}
\title{Rapport du projet de Recherche Opérationnelle} 
\author{NOBLET Yvan, GIBAUD Lary}

\begin{document}

\date{}
\maketitle
\tableofcontents
\section{Introduction}

Ceci est le rapport du projet de recherche opérationnelle : 
\emph{Une méthode simple pour la résolution exacte d’un problème
 de tournée de véhicule avec capacité}.

Le but est de fournir un programme permettant de définir un plan de livraison de clients par des drones. 
Ce but est rempli par un programme écrit en langage C et l'utilisation de la bibliothèque GLPK.

\section{Analyse}

\subsection{Squelette du programme}

Le squelette du programme C nous est donné. Celui-ci contient les inclusions de librairies nécessaires pour faire fonctionner GLPK ainsi que le parser permettant de lire les fichiers de données.

\subsection{Fichiers de données}

Deux jeux de données nous sont fournis (les répertoires A et B) ainsi que l'exemple de l'énoncé (exemple.dat). Le format du fichier de données est ainsi :
\begin{verbatim}
ligne 1 est : 
ligne 2 est :
les lignes suivantes sont :
\end{verbatim}

\subsection{Ce qui est demandé}

Soit $\Omega$ un ensemble de numéros de clients : $\Omega = {1,2,...,n}$.

\end{exemple}
\end{document}
